\documentclass[a4paper]{amsart}            %for bookmarks enable option [liststotoc]%



%-------packages-------------------------%


\usepackage{amsmath}
\usepackage{amsthm}        % Does theorem stuff
\usepackage{amssymb} 
\usepackage{xypic}  				%for strange reason I need this to make the two cell diagrams				
\usepackage[2cell]{xy} 			%for commutative diagrams%

\usepackage{color,hyperref}
\definecolor{darkblue}{rgb}{0.0,0.0,0.3}                      %pdf book marks the way I like%
\hypersetup{pdftex=true, colorlinks=true, breaklinks=true, linkcolor=darkblue, menucolor=darkblue, pagecolor=darkblue, urlcolor=darkblue}



%-----------style------------------------%

\addtolength{\parskip}{\baselineskip} %Abs�tze im Text werden auch tats�chlich zu Abs�tzen%
\parindent 0pt

%----------new--commands-----------------%

\newcommand{\C}{\mathcal{C}}
\newcommand{\D}{\mathcal{D}}
\newcommand{\F}{\mathcal{F}}
\newcommand{\G}{\mathcal{G}}
\newcommand{\ve}{\varepsilon}
\newcommand{\Mor}{\mathrm{Mor}}
\newcommand{\id}{\operatorname{id}}
\newcommand{\Hom}{\operatorname{Hom}}
%this command was proposed by Stefan_K%
\newcommand*{\leftterm}{\left\{\substack{\normalsize \text{natural transformations}
  \\h^A \rightarrow F} \right\}}
\newlength{\leftside}
\settowidth{\leftside}{$\leftterm$}

% the following code will uplift the \maketitle title. In standard it is way too low.
\makeatletter % wegen @ in den Befehsnamen
\renewcommand*\@maketitle{%
  \normalfont\normalsize
  \@adminfootnotes
  \@mkboth{\@nx\shortauthors}{\@nx\shorttitle}%
% (SCHW) auskommentiert:  \global\topskip42\p@\relax % 5.5pc   "   "   "     "     "
  \@settitle
  \ifx\@empty\authors \else \@setauthors \fi
  \ifx\@empty\@dedicatory
  \else
    \baselineskip18\p@
    \vtop{\centering{\footnotesize\itshape\@dedicatory\@@par}%
      \global\dimen@i\prevdepth}\prevdepth\dimen@i
  \fi
  \@setabstract
  \normalsize
  \if@titlepage
    \newpage
  \else
    \dimen@34\p@ \advance\dimen@-\baselineskip
    \vskip\dimen@\relax
  \fi
} % end \@maketitle
\makeatother

%--------new--enviroments----------------%

\renewcommand{\theequation}{\thesection.\arabic{equation}}         %%                             %I somehow need this for the warning symbol to work 
 \makeatletter                                                      %%                             properly, I don't really know why though%
    \@addtoreset{equation}{section} % Make the equation counter reset each section
    \@addtoreset{footnote}{section} % Make the footnote counter reset each section
                                                                    %%
 \newenvironment{warning}[1][]{%                                    %%
    \begin{trivlist} \item[] \noindent%                             %%
    \begingroup\hangindent=2pc\hangafter=-2                         
    \clubpenalty=10000%                                             											This is my current proof environment look below for another version
    \hbox to0pt{\hskip-\hangindent\manfntsymbol{127}\hfill}\ignorespaces%
    \refstepcounter{equation}\textbf{Warning~\theequation}%         
    \@ifnotempty{#1}{\the\thm@notefont \ (#1)}\textbf{.}            
    \let\p@@r=\par \def\p@r{\p@@r \hangindent=0pc} \let\par=\p@r}%  
    {\hspace*{\fill}$\lrcorner$\endgraf\endgroup\end{trivlist}}  
    
    
%\newenvironment{beweis}{\par\begingroup%
%\settowidth{\leftskip}{\emph{Proof.~}}%																											remove to use \begin{beweis}
%\noindent\llap{\emph{Proof.~}}}{\hfill$\Box$\par\endgroup}     
    
  

\renewenvironment{proof}[1][\proofname]{\par
  \pushQED{\qed}%
  \normalfont 
  \trivlist
  \leftskip=0.5cm
  \item[\hskip\labelsep
        \itshape
    #1\@{.}]\mbox{ }\par\noindent%
  \leftskip=1cm
  \rightskip=1cm
}{%
  \popQED\endtrivlist\unhbox \voidb@x\@}    

\makeatother
      

\theoremstyle{plain}                                               %%
 \newtheorem{theorem}[equation]{Theorem}                            %%
 \newtheorem*{claim}{Claim}                                         %%
 \newtheorem*{lemma*}{Lemma}                                        %%
 \newtheorem*{theorem*}{Theorem}                                    %%
 \newtheorem{lemma}[equation]{Lemma}                                %%
 \newtheorem{corollary}[equation]{Corollary}                        %%
 \newtheorem{proposition}[equation]{Proposition}                    %%

%--produce-the-warning--symbol-----------%
 \DeclareFontFamily{U}{manual}{}                                
 \DeclareFontShape{U}{manual}{m}{n}{ <->  manfnt }{}            
 \newcommand{\manfntsymbol}[1]{%                                
    {\fontencoding{U}\fontfamily{manual}\selectfont\symbol{#1}}}

\makeatother


\begin{document}

%------------header----------------------%

\title{Yoneda lemma}
%\author{Felix Hoffmann}% %for now I decided against it%


\maketitle
\pagestyle{empty} %no pagenumbers and titles%
\thispagestyle{empty}


%-------end header-----------------------%

Given a category $\C$, we can form a category $\textbf{Set}^{\C}$, consisting of all functors ${\C \rightarrow \textbf{Set}}$ with natural transformations 
as morphisms. The composition of morphisms in $\textbf{Set}^{\C}$ is then just the vertical composition of natural transformations.

One easily sees that this composition is asscociative and for all functors ${F: \C \rightarrow \textbf{Set}}$ the natural transformation $F \rightarrow F$
with components $\id_{C}$ is an identity, thus we have indeed a category. We call $\textbf{Set}^{\C}$ the \emph{functor category} of functors from
$\C$ to $\textbf{Set}$.

For each object $A$ in $\C$ we obtain a special functor ${\C \rightarrow \textbf{Set}}$ called the \emph{Hom-functor} of $A$. It sends each object $X$
in $\C$ to the Hom-Set $\Mor_{\C}$.

\begin{align*}h^A: &&\mathcal{C} 			           	      &\longrightarrow \textbf{Set} \\      
     &&X 								                 	&\longmapsto \Mor_{\C}(A,X)   \\
     &&\left(X \xrightarrow{f} Y\right) 	&\longmapsto \left({\begin{aligned} \Mor_{\C}(A,X) &\xrightarrow{h^A(f)} \Mor_{\C}(A,Y)	\\
																								  	                               A \to X  &\longmapsto	\xymatrix@=+0.1cm{A \ar[rr] \ar[rdd] & 				      & Y \\
																												 							 																		                                     &              &   \\
																												 																										                                   &X \ar[uur]^{f}&     }
                                                              \end{aligned}}\right)\end{align*}

In order to secure that the Hom-sets are indeed sets, we demand that $\C$ is \emph{locally small}.

\textbf{Yoneda's lemma} now states that there is a bijection between the set of natural transformations $h^A \rightarrow F$ and the set $F(A)$ for any 
functor $F: \C \rightarrow \textbf{Set}$.%
%
\begin{proof}%
Given a natural transformation $h^A \rightarrow F$ with components $m_C$ we define the map
%
\begin{align*}f: \leftterm &\longrightarrow F(A) \\
  \makebox[\leftside]{$h^A \rightarrow F$} &\longmapsto m_A(\id_A)\end{align*}
%
Obviously it is well defined, we now check for surjectivity and injectivity.

For $c \in F(A)$ we just define a natural transformation by assigning $\id_A \mapsto c$ via the component $m_A$. 
This not only gives a natural transformation - it is actually completely determined by that assignment: 

To see this consider any object $X$ in $\C$. We need to give a map 
\begin{equation*}
{h^A(X)=\Mor (A,X) \rightarrow F(X)}.
\end{equation*}
If $\Mor (A,X)$ is empty, there there is only one possible choice. Else let $g \in \Mor (A,X)$. Then we consider the diagram
%
%
\begin{equation*}
\xymatrix @+=1.55cm{\Mor (A,A) \ar[r]^{g \circ -} \ar[d]_{m_A} & \Mor(A,X) \ar[d]^{m_X} \\
					F(A) \ar[r]^{F(g)} 															 & F(X)                     }
\end{equation*}
%
%
in which we don't know the vertical arrows, but we \emph{do} know two things:
%
\begin{equation*}
\id_A \mapsto m_A(\id_A)=c 
\end{equation*}
%
and that the diagram has to constitute a naturality square.

Thus we get, by tracking $\id_A$,

\begin{equation*}
\boxed{m_X(g \circ \id_A)=F(g) \circ m_A(\id_A)}.
\end{equation*}

Thus $m_X(g)$ is completely determined by $m_A(\id_A)$.
\end{proof}

\begin{lemma}
	Let $F:\C \to \D$ be fully faithful functor. Then, whenever $F(X) \xrightarrow{F(f)} F(X')$ is an isomorphism in $\D$, $X \xrightarrow{f} X'$ is an isomorphism in $\C$.
\end{lemma}

\medskip

\begin{corollary}
	Fully faithful functors are essentially injective.
\end{corollary}

\begin{proof}%
	Let $F(X) \xrightarrow{F(f)} F(X')$ be an isomorphism. Then there exists 
	\vspace{0.2cm}
	\begin{equation*}
			\varphi \in \Mor_\D(F(X'),F(X))
		\vspace{0.2cm}
	\end{equation*}	
	such that $F(f) \circ \varphi=\id_{F(X')}$, $\varphi \circ F(f)=\id_{F(X)}$. 
	
	Since $F$ is fully faithful there exists a morphism $g \in \Mor(X,X')$, 
	such that $F(g)=\varphi$. 
	(This is the injectivity of the map $\Mor(X,X') \to \Mor(F(X),F(X'))$.)
	
	Then we have 
	\vspace{0.2cm}
	\begin{equation*}
		F(g \circ f)=F(g) \circ F(f)=\id_{F(X)}=F(\id_X).
	\vspace{0.2cm}
	\end{equation*}
	Injectivity then gives $g \circ f=\id_X$ amd similarily $f \circ g=\id_{X'}$.%
\end{proof}

In summary one can say:

\emph{Fully faithful functors reflect isomorphisms.}




\end{document}